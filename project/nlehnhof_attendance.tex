\documentclass[conf]{new-aiaa}

\input{preamble}

\begin{document}
\sloppy

\title{The Effects of Absences on Student Performance}
\author{Nathan R. Lehnhof\footnote{Mechanical Engineering}}
\affil{BYU Provo, UT}
\date{\today}
\maketitle

% ======================================================
\begin{abstract}
Student attendance is widely recognized as a critical factor influencing academic success. 
Absences can disrupt learning continuity, reduce exposure to instructional materials, and impair performance on assessments. 
In educational research, frequent absences have been linked to lower achievement, decreased engagement, and increased risk of academic difficulties.
This report investigates whether patterns of student absences support this understanding and quantifies the relationship between absenteeism and academic performance. 

Using a random sample of Kaggle's Students Performance Dataset, we compare test scores between students with fewer than five absences and those with five or more. 
A two-sample t-test is conducted to determine whether the difference in mean performance is statistically significant, followed by development of a confidence interval for the mean difference.
The results allow us to evaluate commonly held beliefs about attendance and achievement.
\end{abstract}

% ======================================================
\section{Introduction}
Regular classroom attendance plays an essential role in supporting academic success. 
Consistent presence allows students to receive instruction, participate in learning activities, and develop understanding through repeated engagement. 
Numerous studies indicate that absenteeism negatively affects learning outcomes and increases the likelihood of academic struggle.

However, students may not always recognize the long-term effects of missing class. 
Some may assume that occasional absences have little impact or that material can be made up independently. 
Although prior research suggests that absenteeism reduces performance, the magnitude of its effect can vary across students, schools, and subjects.

This project examines this question using an openly available dataset of student test scores. 
We analyze data from Kaggle's Student Performance Database to explore whether students with higher numbers of absences perform worse academically. 
By statistically comparing performance between students with fewer than five absences and those with five or more, we aim to determine whether this widely held belief is supported by evidence in a real-world sample.

The goal of this study is not only to evaluate a common assumption but also to quantify the performance differences associated with absenteeism. 
This analysis allows us to assess whether increased absence corresponds to meaningful differences in academic outcomes and whether the relationship is statistically significant under formal hypothesis testing.

% ======================================================
\section{Methods}

% ======================================================
\section{Results}

\FloatBarrier

% ======================================================
\section{Conclusion}

\section{References and Resources}
\bibliography{references}

\end{document}
